\documentclass{article}

\usepackage[ngerman]{babel}
\usepackage[utf8]{inputenc}
\usepackage[T1]{fontenc}
\usepackage{hyperref}
\usepackage{csquotes}

\usepackage[
    backend=biber,
    style=apa,
    sortlocale=de_DE,
    natbib=true,
    url=false,
    doi=false,
    sortcites=true,
    sorting=nyt,
    isbn=false,
    hyperref=true,
    backref=false,
    giveninits=false,
    eprint=false]{biblatex}
\addbibresource{../references/bibliography.bib}

\title{Notizen zum Projekt Data Ethics}
\author{Luka Adjancic}
\date{\today}

\begin{document}
\maketitle

\abstract{
    Dieses Dokument ist eine Sammlung von Notizen zu dem Projekt. Die Struktur innerhalb des
    Projektes ist gleich ausgelegt wie in der Hauptarbeit, somit kann hier einfach geschrieben
    werden, und die Teile die man verwenden möchte, kann man direkt in die Hauptdatei ziehen.
}

\tableofcontents

\section{Künstliche Intelligenz}
\label{sec:ai}
\textbf{Was ist ki?} \newline
Künstliche Intelligenz ist ein Teilgebiet der Informatik, das sich mit der Entwicklung von Computersystemen befasst, die Aufgaben erfüllen können, die normalerweise menschliche Intelligenz erfordern.

\section{Ethik}
\label{sec:ethic}


\section{Daten}
\label{sec:data}
\textbf{Was sind Daten?}\newline
Daten sind Informationen, die in eine Form übersetzt wurden, die für das Kopieren oder Verarbeiten effizient ist. In heutigen Computern und Übertragungsmedien sind Daten Informationen, die in eine binäre digitale Form umgewandelt wurden.\newline
\textbf{Wie sind Daten mit KI verknüpft?}\newline
Daten sind mit KI verknüpft, da automatisierte/künstliche Intelligenz (KI) Handlungen und Entscheidungen basierend auf Daten im Allgemeinen und personenbezogenen Daten im Besonderen steuert. Algorithmen für KI und maschinelles Lernen lernen aus den Rückmeldungen der Nutzer basierend auf Trainingsdaten, die Verzerrungen enthalten können. Es ist wichtig, dass Daten, Algorithmen und Technologien fair und vorurteilsfrei genutzt werden.\newline
\textbf{Welche Daten werden geschützt?}\newline
Die Arten von Daten, die durch die DSGVO geschützt sind, sind personenbezogene Daten wie Namen, Adressen, Telefonnummern, E-Mail-Adressen, etc. Diese Daten sind geschützt, um die Privatsphäre und die Rechte der betroffenen Personen zu wahren. Beispiele für geschützte Daten sind Kundendaten, Mitarbeiterdaten und Lieferantendaten. Es ist wichtig, diese Datenethik zu beachten, um das Vertrauen der Kunden zu gewinnen und zu erhalten. Durch Maßnahmen wie Datenverschlüsselung und Datenintegrität können Risiken wie Ransomware und Datenmanipulation minimiert werden, was wiederum das Vertrauen der Kunden stärkt.\newline
\textbf{Wie werden Daten verkauft?}\newline
Daten werden von Datenhändlern an andere Unternehmen verkauft, die sie für kommerzielle Zwecke nutzen.

\printbibliography

\end{document}
